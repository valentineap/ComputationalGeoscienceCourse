
% Default to the notebook output style

    


% Inherit from the specified cell style.




    
\documentclass[11pt]{article}

    
    
    \usepackage[T1]{fontenc}
    % Nicer default font (+ math font) than Computer Modern for most use cases
    \usepackage{mathpazo}

    % Basic figure setup, for now with no caption control since it's done
    % automatically by Pandoc (which extracts ![](path) syntax from Markdown).
    \usepackage{graphicx}
    % We will generate all images so they have a width \maxwidth. This means
    % that they will get their normal width if they fit onto the page, but
    % are scaled down if they would overflow the margins.
    \makeatletter
    \def\maxwidth{\ifdim\Gin@nat@width>\linewidth\linewidth
    \else\Gin@nat@width\fi}
    \makeatother
    \let\Oldincludegraphics\includegraphics
    % Set max figure width to be 80% of text width, for now hardcoded.
    \renewcommand{\includegraphics}[1]{\Oldincludegraphics[width=.8\maxwidth]{#1}}
    % Ensure that by default, figures have no caption (until we provide a
    % proper Figure object with a Caption API and a way to capture that
    % in the conversion process - todo).
    \usepackage{caption}
    \DeclareCaptionLabelFormat{nolabel}{}
    \captionsetup{labelformat=nolabel}

    \usepackage{adjustbox} % Used to constrain images to a maximum size 
    \usepackage{xcolor} % Allow colors to be defined
    \usepackage{enumerate} % Needed for markdown enumerations to work
    \usepackage{geometry} % Used to adjust the document margins
    \usepackage{amsmath} % Equations
    \usepackage{amssymb} % Equations
    \usepackage{textcomp} % defines textquotesingle
    % Hack from http://tex.stackexchange.com/a/47451/13684:
    \AtBeginDocument{%
        \def\PYZsq{\textquotesingle}% Upright quotes in Pygmentized code
    }
    \usepackage{upquote} % Upright quotes for verbatim code
    \usepackage{eurosym} % defines \euro
    \usepackage[mathletters]{ucs} % Extended unicode (utf-8) support
    \usepackage[utf8x]{inputenc} % Allow utf-8 characters in the tex document
    \usepackage{fancyvrb} % verbatim replacement that allows latex
    \usepackage{grffile} % extends the file name processing of package graphics 
                         % to support a larger range 
    % The hyperref package gives us a pdf with properly built
    % internal navigation ('pdf bookmarks' for the table of contents,
    % internal cross-reference links, web links for URLs, etc.)
    \usepackage{hyperref}
    \usepackage{longtable} % longtable support required by pandoc >1.10
    \usepackage{booktabs}  % table support for pandoc > 1.12.2
    \usepackage[inline]{enumitem} % IRkernel/repr support (it uses the enumerate* environment)
    \usepackage[normalem]{ulem} % ulem is needed to support strikethroughs (\sout)
                                % normalem makes italics be italics, not underlines
    \usepackage{mathrsfs}
    

    
    
    % Colors for the hyperref package
    \definecolor{urlcolor}{rgb}{0,.145,.698}
    \definecolor{linkcolor}{rgb}{.71,0.21,0.01}
    \definecolor{citecolor}{rgb}{.12,.54,.11}

    % ANSI colors
    \definecolor{ansi-black}{HTML}{3E424D}
    \definecolor{ansi-black-intense}{HTML}{282C36}
    \definecolor{ansi-red}{HTML}{E75C58}
    \definecolor{ansi-red-intense}{HTML}{B22B31}
    \definecolor{ansi-green}{HTML}{00A250}
    \definecolor{ansi-green-intense}{HTML}{007427}
    \definecolor{ansi-yellow}{HTML}{DDB62B}
    \definecolor{ansi-yellow-intense}{HTML}{B27D12}
    \definecolor{ansi-blue}{HTML}{208FFB}
    \definecolor{ansi-blue-intense}{HTML}{0065CA}
    \definecolor{ansi-magenta}{HTML}{D160C4}
    \definecolor{ansi-magenta-intense}{HTML}{A03196}
    \definecolor{ansi-cyan}{HTML}{60C6C8}
    \definecolor{ansi-cyan-intense}{HTML}{258F8F}
    \definecolor{ansi-white}{HTML}{C5C1B4}
    \definecolor{ansi-white-intense}{HTML}{A1A6B2}
    \definecolor{ansi-default-inverse-fg}{HTML}{FFFFFF}
    \definecolor{ansi-default-inverse-bg}{HTML}{000000}

    % commands and environments needed by pandoc snippets
    % extracted from the output of `pandoc -s`
    \providecommand{\tightlist}{%
      \setlength{\itemsep}{0pt}\setlength{\parskip}{0pt}}
    \DefineVerbatimEnvironment{Highlighting}{Verbatim}{commandchars=\\\{\}}
    % Add ',fontsize=\small' for more characters per line
    \newenvironment{Shaded}{}{}
    \newcommand{\KeywordTok}[1]{\textcolor[rgb]{0.00,0.44,0.13}{\textbf{{#1}}}}
    \newcommand{\DataTypeTok}[1]{\textcolor[rgb]{0.56,0.13,0.00}{{#1}}}
    \newcommand{\DecValTok}[1]{\textcolor[rgb]{0.25,0.63,0.44}{{#1}}}
    \newcommand{\BaseNTok}[1]{\textcolor[rgb]{0.25,0.63,0.44}{{#1}}}
    \newcommand{\FloatTok}[1]{\textcolor[rgb]{0.25,0.63,0.44}{{#1}}}
    \newcommand{\CharTok}[1]{\textcolor[rgb]{0.25,0.44,0.63}{{#1}}}
    \newcommand{\StringTok}[1]{\textcolor[rgb]{0.25,0.44,0.63}{{#1}}}
    \newcommand{\CommentTok}[1]{\textcolor[rgb]{0.38,0.63,0.69}{\textit{{#1}}}}
    \newcommand{\OtherTok}[1]{\textcolor[rgb]{0.00,0.44,0.13}{{#1}}}
    \newcommand{\AlertTok}[1]{\textcolor[rgb]{1.00,0.00,0.00}{\textbf{{#1}}}}
    \newcommand{\FunctionTok}[1]{\textcolor[rgb]{0.02,0.16,0.49}{{#1}}}
    \newcommand{\RegionMarkerTok}[1]{{#1}}
    \newcommand{\ErrorTok}[1]{\textcolor[rgb]{1.00,0.00,0.00}{\textbf{{#1}}}}
    \newcommand{\NormalTok}[1]{{#1}}
    
    % Additional commands for more recent versions of Pandoc
    \newcommand{\ConstantTok}[1]{\textcolor[rgb]{0.53,0.00,0.00}{{#1}}}
    \newcommand{\SpecialCharTok}[1]{\textcolor[rgb]{0.25,0.44,0.63}{{#1}}}
    \newcommand{\VerbatimStringTok}[1]{\textcolor[rgb]{0.25,0.44,0.63}{{#1}}}
    \newcommand{\SpecialStringTok}[1]{\textcolor[rgb]{0.73,0.40,0.53}{{#1}}}
    \newcommand{\ImportTok}[1]{{#1}}
    \newcommand{\DocumentationTok}[1]{\textcolor[rgb]{0.73,0.13,0.13}{\textit{{#1}}}}
    \newcommand{\AnnotationTok}[1]{\textcolor[rgb]{0.38,0.63,0.69}{\textbf{\textit{{#1}}}}}
    \newcommand{\CommentVarTok}[1]{\textcolor[rgb]{0.38,0.63,0.69}{\textbf{\textit{{#1}}}}}
    \newcommand{\VariableTok}[1]{\textcolor[rgb]{0.10,0.09,0.49}{{#1}}}
    \newcommand{\ControlFlowTok}[1]{\textcolor[rgb]{0.00,0.44,0.13}{\textbf{{#1}}}}
    \newcommand{\OperatorTok}[1]{\textcolor[rgb]{0.40,0.40,0.40}{{#1}}}
    \newcommand{\BuiltInTok}[1]{{#1}}
    \newcommand{\ExtensionTok}[1]{{#1}}
    \newcommand{\PreprocessorTok}[1]{\textcolor[rgb]{0.74,0.48,0.00}{{#1}}}
    \newcommand{\AttributeTok}[1]{\textcolor[rgb]{0.49,0.56,0.16}{{#1}}}
    \newcommand{\InformationTok}[1]{\textcolor[rgb]{0.38,0.63,0.69}{\textbf{\textit{{#1}}}}}
    \newcommand{\WarningTok}[1]{\textcolor[rgb]{0.38,0.63,0.69}{\textbf{\textit{{#1}}}}}
    
    
    % Define a nice break command that doesn't care if a line doesn't already
    % exist.
    \def\br{\hspace*{\fill} \\* }
    % Math Jax compatibility definitions
    \def\gt{>}
    \def\lt{<}
    \let\Oldtex\TeX
    \let\Oldlatex\LaTeX
    \renewcommand{\TeX}{\textrm{\Oldtex}}
    \renewcommand{\LaTeX}{\textrm{\Oldlatex}}
    % Document parameters
    % Document title
    \title{EMSC8014 Assessment}
    
    
    
    
    

    % Pygments definitions
    
\makeatletter
\def\PY@reset{\let\PY@it=\relax \let\PY@bf=\relax%
    \let\PY@ul=\relax \let\PY@tc=\relax%
    \let\PY@bc=\relax \let\PY@ff=\relax}
\def\PY@tok#1{\csname PY@tok@#1\endcsname}
\def\PY@toks#1+{\ifx\relax#1\empty\else%
    \PY@tok{#1}\expandafter\PY@toks\fi}
\def\PY@do#1{\PY@bc{\PY@tc{\PY@ul{%
    \PY@it{\PY@bf{\PY@ff{#1}}}}}}}
\def\PY#1#2{\PY@reset\PY@toks#1+\relax+\PY@do{#2}}

\expandafter\def\csname PY@tok@w\endcsname{\def\PY@tc##1{\textcolor[rgb]{0.73,0.73,0.73}{##1}}}
\expandafter\def\csname PY@tok@c\endcsname{\let\PY@it=\textit\def\PY@tc##1{\textcolor[rgb]{0.25,0.50,0.50}{##1}}}
\expandafter\def\csname PY@tok@cp\endcsname{\def\PY@tc##1{\textcolor[rgb]{0.74,0.48,0.00}{##1}}}
\expandafter\def\csname PY@tok@k\endcsname{\let\PY@bf=\textbf\def\PY@tc##1{\textcolor[rgb]{0.00,0.50,0.00}{##1}}}
\expandafter\def\csname PY@tok@kp\endcsname{\def\PY@tc##1{\textcolor[rgb]{0.00,0.50,0.00}{##1}}}
\expandafter\def\csname PY@tok@kt\endcsname{\def\PY@tc##1{\textcolor[rgb]{0.69,0.00,0.25}{##1}}}
\expandafter\def\csname PY@tok@o\endcsname{\def\PY@tc##1{\textcolor[rgb]{0.40,0.40,0.40}{##1}}}
\expandafter\def\csname PY@tok@ow\endcsname{\let\PY@bf=\textbf\def\PY@tc##1{\textcolor[rgb]{0.67,0.13,1.00}{##1}}}
\expandafter\def\csname PY@tok@nb\endcsname{\def\PY@tc##1{\textcolor[rgb]{0.00,0.50,0.00}{##1}}}
\expandafter\def\csname PY@tok@nf\endcsname{\def\PY@tc##1{\textcolor[rgb]{0.00,0.00,1.00}{##1}}}
\expandafter\def\csname PY@tok@nc\endcsname{\let\PY@bf=\textbf\def\PY@tc##1{\textcolor[rgb]{0.00,0.00,1.00}{##1}}}
\expandafter\def\csname PY@tok@nn\endcsname{\let\PY@bf=\textbf\def\PY@tc##1{\textcolor[rgb]{0.00,0.00,1.00}{##1}}}
\expandafter\def\csname PY@tok@ne\endcsname{\let\PY@bf=\textbf\def\PY@tc##1{\textcolor[rgb]{0.82,0.25,0.23}{##1}}}
\expandafter\def\csname PY@tok@nv\endcsname{\def\PY@tc##1{\textcolor[rgb]{0.10,0.09,0.49}{##1}}}
\expandafter\def\csname PY@tok@no\endcsname{\def\PY@tc##1{\textcolor[rgb]{0.53,0.00,0.00}{##1}}}
\expandafter\def\csname PY@tok@nl\endcsname{\def\PY@tc##1{\textcolor[rgb]{0.63,0.63,0.00}{##1}}}
\expandafter\def\csname PY@tok@ni\endcsname{\let\PY@bf=\textbf\def\PY@tc##1{\textcolor[rgb]{0.60,0.60,0.60}{##1}}}
\expandafter\def\csname PY@tok@na\endcsname{\def\PY@tc##1{\textcolor[rgb]{0.49,0.56,0.16}{##1}}}
\expandafter\def\csname PY@tok@nt\endcsname{\let\PY@bf=\textbf\def\PY@tc##1{\textcolor[rgb]{0.00,0.50,0.00}{##1}}}
\expandafter\def\csname PY@tok@nd\endcsname{\def\PY@tc##1{\textcolor[rgb]{0.67,0.13,1.00}{##1}}}
\expandafter\def\csname PY@tok@s\endcsname{\def\PY@tc##1{\textcolor[rgb]{0.73,0.13,0.13}{##1}}}
\expandafter\def\csname PY@tok@sd\endcsname{\let\PY@it=\textit\def\PY@tc##1{\textcolor[rgb]{0.73,0.13,0.13}{##1}}}
\expandafter\def\csname PY@tok@si\endcsname{\let\PY@bf=\textbf\def\PY@tc##1{\textcolor[rgb]{0.73,0.40,0.53}{##1}}}
\expandafter\def\csname PY@tok@se\endcsname{\let\PY@bf=\textbf\def\PY@tc##1{\textcolor[rgb]{0.73,0.40,0.13}{##1}}}
\expandafter\def\csname PY@tok@sr\endcsname{\def\PY@tc##1{\textcolor[rgb]{0.73,0.40,0.53}{##1}}}
\expandafter\def\csname PY@tok@ss\endcsname{\def\PY@tc##1{\textcolor[rgb]{0.10,0.09,0.49}{##1}}}
\expandafter\def\csname PY@tok@sx\endcsname{\def\PY@tc##1{\textcolor[rgb]{0.00,0.50,0.00}{##1}}}
\expandafter\def\csname PY@tok@m\endcsname{\def\PY@tc##1{\textcolor[rgb]{0.40,0.40,0.40}{##1}}}
\expandafter\def\csname PY@tok@gh\endcsname{\let\PY@bf=\textbf\def\PY@tc##1{\textcolor[rgb]{0.00,0.00,0.50}{##1}}}
\expandafter\def\csname PY@tok@gu\endcsname{\let\PY@bf=\textbf\def\PY@tc##1{\textcolor[rgb]{0.50,0.00,0.50}{##1}}}
\expandafter\def\csname PY@tok@gd\endcsname{\def\PY@tc##1{\textcolor[rgb]{0.63,0.00,0.00}{##1}}}
\expandafter\def\csname PY@tok@gi\endcsname{\def\PY@tc##1{\textcolor[rgb]{0.00,0.63,0.00}{##1}}}
\expandafter\def\csname PY@tok@gr\endcsname{\def\PY@tc##1{\textcolor[rgb]{1.00,0.00,0.00}{##1}}}
\expandafter\def\csname PY@tok@ge\endcsname{\let\PY@it=\textit}
\expandafter\def\csname PY@tok@gs\endcsname{\let\PY@bf=\textbf}
\expandafter\def\csname PY@tok@gp\endcsname{\let\PY@bf=\textbf\def\PY@tc##1{\textcolor[rgb]{0.00,0.00,0.50}{##1}}}
\expandafter\def\csname PY@tok@go\endcsname{\def\PY@tc##1{\textcolor[rgb]{0.53,0.53,0.53}{##1}}}
\expandafter\def\csname PY@tok@gt\endcsname{\def\PY@tc##1{\textcolor[rgb]{0.00,0.27,0.87}{##1}}}
\expandafter\def\csname PY@tok@err\endcsname{\def\PY@bc##1{\setlength{\fboxsep}{0pt}\fcolorbox[rgb]{1.00,0.00,0.00}{1,1,1}{\strut ##1}}}
\expandafter\def\csname PY@tok@kc\endcsname{\let\PY@bf=\textbf\def\PY@tc##1{\textcolor[rgb]{0.00,0.50,0.00}{##1}}}
\expandafter\def\csname PY@tok@kd\endcsname{\let\PY@bf=\textbf\def\PY@tc##1{\textcolor[rgb]{0.00,0.50,0.00}{##1}}}
\expandafter\def\csname PY@tok@kn\endcsname{\let\PY@bf=\textbf\def\PY@tc##1{\textcolor[rgb]{0.00,0.50,0.00}{##1}}}
\expandafter\def\csname PY@tok@kr\endcsname{\let\PY@bf=\textbf\def\PY@tc##1{\textcolor[rgb]{0.00,0.50,0.00}{##1}}}
\expandafter\def\csname PY@tok@bp\endcsname{\def\PY@tc##1{\textcolor[rgb]{0.00,0.50,0.00}{##1}}}
\expandafter\def\csname PY@tok@fm\endcsname{\def\PY@tc##1{\textcolor[rgb]{0.00,0.00,1.00}{##1}}}
\expandafter\def\csname PY@tok@vc\endcsname{\def\PY@tc##1{\textcolor[rgb]{0.10,0.09,0.49}{##1}}}
\expandafter\def\csname PY@tok@vg\endcsname{\def\PY@tc##1{\textcolor[rgb]{0.10,0.09,0.49}{##1}}}
\expandafter\def\csname PY@tok@vi\endcsname{\def\PY@tc##1{\textcolor[rgb]{0.10,0.09,0.49}{##1}}}
\expandafter\def\csname PY@tok@vm\endcsname{\def\PY@tc##1{\textcolor[rgb]{0.10,0.09,0.49}{##1}}}
\expandafter\def\csname PY@tok@sa\endcsname{\def\PY@tc##1{\textcolor[rgb]{0.73,0.13,0.13}{##1}}}
\expandafter\def\csname PY@tok@sb\endcsname{\def\PY@tc##1{\textcolor[rgb]{0.73,0.13,0.13}{##1}}}
\expandafter\def\csname PY@tok@sc\endcsname{\def\PY@tc##1{\textcolor[rgb]{0.73,0.13,0.13}{##1}}}
\expandafter\def\csname PY@tok@dl\endcsname{\def\PY@tc##1{\textcolor[rgb]{0.73,0.13,0.13}{##1}}}
\expandafter\def\csname PY@tok@s2\endcsname{\def\PY@tc##1{\textcolor[rgb]{0.73,0.13,0.13}{##1}}}
\expandafter\def\csname PY@tok@sh\endcsname{\def\PY@tc##1{\textcolor[rgb]{0.73,0.13,0.13}{##1}}}
\expandafter\def\csname PY@tok@s1\endcsname{\def\PY@tc##1{\textcolor[rgb]{0.73,0.13,0.13}{##1}}}
\expandafter\def\csname PY@tok@mb\endcsname{\def\PY@tc##1{\textcolor[rgb]{0.40,0.40,0.40}{##1}}}
\expandafter\def\csname PY@tok@mf\endcsname{\def\PY@tc##1{\textcolor[rgb]{0.40,0.40,0.40}{##1}}}
\expandafter\def\csname PY@tok@mh\endcsname{\def\PY@tc##1{\textcolor[rgb]{0.40,0.40,0.40}{##1}}}
\expandafter\def\csname PY@tok@mi\endcsname{\def\PY@tc##1{\textcolor[rgb]{0.40,0.40,0.40}{##1}}}
\expandafter\def\csname PY@tok@il\endcsname{\def\PY@tc##1{\textcolor[rgb]{0.40,0.40,0.40}{##1}}}
\expandafter\def\csname PY@tok@mo\endcsname{\def\PY@tc##1{\textcolor[rgb]{0.40,0.40,0.40}{##1}}}
\expandafter\def\csname PY@tok@ch\endcsname{\let\PY@it=\textit\def\PY@tc##1{\textcolor[rgb]{0.25,0.50,0.50}{##1}}}
\expandafter\def\csname PY@tok@cm\endcsname{\let\PY@it=\textit\def\PY@tc##1{\textcolor[rgb]{0.25,0.50,0.50}{##1}}}
\expandafter\def\csname PY@tok@cpf\endcsname{\let\PY@it=\textit\def\PY@tc##1{\textcolor[rgb]{0.25,0.50,0.50}{##1}}}
\expandafter\def\csname PY@tok@c1\endcsname{\let\PY@it=\textit\def\PY@tc##1{\textcolor[rgb]{0.25,0.50,0.50}{##1}}}
\expandafter\def\csname PY@tok@cs\endcsname{\let\PY@it=\textit\def\PY@tc##1{\textcolor[rgb]{0.25,0.50,0.50}{##1}}}

\def\PYZbs{\char`\\}
\def\PYZus{\char`\_}
\def\PYZob{\char`\{}
\def\PYZcb{\char`\}}
\def\PYZca{\char`\^}
\def\PYZam{\char`\&}
\def\PYZlt{\char`\<}
\def\PYZgt{\char`\>}
\def\PYZsh{\char`\#}
\def\PYZpc{\char`\%}
\def\PYZdl{\char`\$}
\def\PYZhy{\char`\-}
\def\PYZsq{\char`\'}
\def\PYZdq{\char`\"}
\def\PYZti{\char`\~}
% for compatibility with earlier versions
\def\PYZat{@}
\def\PYZlb{[}
\def\PYZrb{]}
\makeatother


    % Exact colors from NB
    \definecolor{incolor}{rgb}{0.0, 0.0, 0.5}
    \definecolor{outcolor}{rgb}{0.545, 0.0, 0.0}



    
    % Prevent overflowing lines due to hard-to-break entities
    \sloppy 
    % Setup hyperref package
    \hypersetup{
      breaklinks=true,  % so long urls are correctly broken across lines
      colorlinks=true,
      urlcolor=urlcolor,
      linkcolor=linkcolor,
      citecolor=citecolor,
      }
    % Slightly bigger margins than the latex defaults
    
    \geometry{verbose,tmargin=1in,bmargin=1in,lmargin=1in,rmargin=1in}
    
    

    \begin{document}
    
    
    \maketitle
    
    

    
    \hypertarget{emsc8014-assessed-exercise}{%
\section*{EMSC8014 Assessed exercise}\label{emsc8014-assessed-exercise}}

\emph{Due: 5pm Friday 26th October}

Once your solution is complete, you should:

\begin{enumerate}
\def\labelenumi{\arabic{enumi}.}
\tightlist
\item
  Ensure it has been saved (select
  \texttt{File\ \textgreater{}\ Save\ and\ checkpoint} from the Jupyter
  menu)
\item
  Select
  \texttt{File\ \textgreater{}\ Download\ as...\ \textgreater{}\ Notebook\ (.ipynb)}
  from the Jupyter menu
\item
  Save the file on your computer
\item
  Submit this file via Wattle.
\end{enumerate}

N.B. Since it is possible to accidentally delete cells from this Jupyter
notebook, you are also provided with a \texttt{.pdf} version. It is your
responsibility to ensure that your solution is complete before
submitting it.

\hypertarget{important-information}{%
\subsubsection*{Important information}\label{important-information}}

\begin{itemize}
\item
  This assessed exercise is made up of a number of tasks. You should
  attempt all of them. The approximate weight to be given to each task
  during grading is shown as a percentage for each.
\item
  You should write your solutions within this Jupyter notebook. You may
  add additional Markdown (text) and code cells as appropriate. Your
  solution should be entirely self-contained (i.e., you should not
  create additional notebooks or other files). Everything appearing in
  the submitted notebook should be intended for assessment; do not
  include rough working.
\item
  Your solution should run without error on the RSES Jupyter server,
  when all cells are executed in the order in which they appear (e.g.~by
  selecting \texttt{Kernel\ \textgreater{}\ Restart\ \&\ Run\ All} from
  the Jupyter menu).
\item
  You may choose to make use of the following modules:

  \begin{itemize}
  \tightlist
  \item
    \texttt{numpy}
  \item
    \texttt{matplotlib}
  \item
    \texttt{pandas}
  \item
    \texttt{cartopy}
  \item
    \texttt{datetime}
  \item
    \texttt{math}
  \item
    \texttt{re}
  \end{itemize}

  \textbf{No other modules may be used} unless you have obtained
  permission from the course convenor (Dr Andrew Valentine). Note that
  you may not need to use all of the modules in the above list.
\item
  Where a task asks you to ``write a function'', you are expected to
  produce one function that can be called to generate all required
  output. However, you are free to organise this as you wish, and you
  may choose to write additional functions which get called within your
  main function.
\item
  You should include an example of how your function is intended to be
  used.
\end{itemize}

\hypertarget{marking-policy}{%
\subsubsection*{Marking policy}\label{marking-policy}}

\begin{itemize}
\item
  There is never a single `correct' way to solve a programming problem.
  You will gain marks for any solution that achieves the desired
  goal(s). Higher marks will be awarded for solutions that are:

  \begin{itemize}
  \tightlist
  \item
    Clear: code is easy for someone else to understand (and perhaps
    modify)
  \item
    Documented: code contains docstrings and comments as appropriate
  \item
    Concise: code is not excessively long or complicated for the task at
    hand
  \item
    General: code can easily be adapted to solve related problems
  \item
    Robust: code appropriately handles any errors that might reasonably
    be expected to occur, especially those related to user input.
  \end{itemize}
\item
  Some tasks require you to produce figures. These should be of a
  standard suitable for inclusion in a research paper. Marks will be
  awarded for:

  \begin{itemize}
  \tightlist
  \item
    Correctness of information presented; remember to include titles,
    axis labels and legends where appropriate.
  \item
    Clarity of layout: is your figure easy to interpret and
    visually-pleasing?
  \item
    Attention to detail: are there any inconsistencies in the way you
    are presenting information?
  \end{itemize}
\item
  If you are unable to complete a task, you should:

  \begin{enumerate}
  \def\labelenumi{\arabic{enumi}.}
  \tightlist
  \item
    Submit your best attempt, and
  \item
    Include a note (comment or Markdown) explaining which piece(s) of
    your solution you believe to be incorrect or incomplete, and why.
  \item
    If you believe you understand the algorithm (sequence of operations)
    to perform a task, but have not been able to implement this in
    Python, you should explain it in full.
  \end{enumerate}

  Partial credit may be given for such work.
\item
  You are allowed to make use of internet resources, class notes, and to
  talk to other people about how to solve these tasks. However,
  \textbf{everything that appears in your solution must be your own,
  individual work}. You should not collaborate with someone else to
  produce a solution, or copy their work. In particular, you
  \textbf{must} understand, and be able to fully explain, everything
  that appears in your solution. \textbf{If you cannot, you will not
  receive any marks for that task.}
\item
  Drop-in sessions (optional) will be held from 12-2 pm on:

  \begin{itemize}
  \tightlist
  \item
    Wednesday 10 October
  \item
    Monday 15 October
  \item
    Friday 19 October
  \item
    Wednesday 24 October
  \end{itemize}

  These will all take place in the TerraView Room at RSES (Jaeger 2, top
  floor). One of the course tutors will be available to answer any
  questions you have.
\item
  Oral exams will be held on Tuesday 30th and Wednesday 31st October, in
  Dr Valentine's office at RSES (Jaeger 2, Room 147a). You will be
  informed of the time at which you should attend. During the oral exam,
  you will be asked to explain your solutions to this assessed exercise,
  and discuss why you chose to implement them in the way that you did.
  No special preparation is required, but you may wish to re-read your
  solutions prior to the examination. \textbf{Failure to attend the oral
  examination (without good cause) will result in you receiving no marks
  for this assessment.}
\end{itemize}

For each task, a \textbf{failing mark} will be awarded where:

\begin{itemize}
\item
  The submitted solution does not accomplish the goal(s) described for
  the task.
\item
  The student is unable to explain how their solution works, and why
  they chose to tackle the problem in a certain way.
\item
  Output (figures/text files) contain significant errors. Figures are
  well below the standard expected for a good-quality journal, through
  omission of important information or poor visual standard.
\item
  The examiners are not satisfied that the solution is the student's
  own, independent work.
\end{itemize}

A \textbf{passing mark} will be awarded where:

\begin{itemize}
\item
  The solution accomplishes the principal goal(s) described for the
  task, perhaps with some minor errors or omissions.
\item
  The code is functional, but unsophisticated; it may contain
  significant redundancy or unnecessary computation.
\item
  It is difficult for someone looking at the code for the first time to
  understand the program logic; variable names are poorly-chosen, and
  the program is not well-organised.
\item
  There is little or no documentation (e.g.~comments and docstrings)
  provided.
\item
  Little or no attempt is made to handle common sources of error
  (e.g.~bad user input).
\item
  Output (figures and/or text files) are accurate but lacking polish.
\item
  The student has demonstrated familiarity with only a limited subset of
  the Python capabilities covered in the course.
\item
  The student is able to explain, in basic terms, how the code solves
  the assigned task.
\end{itemize}

A \textbf{good mark} will be awarded where:

\begin{itemize}
\item
  The solution accomplishes goal(s) described for the task without any
  errors or omissions.
\item
  The code is well-structured and appropriate to the task in hand.
\item
  Code is reasonably understandable, with meaningful variable names and
  a clear organisation.
\item
  Basic documentation is provided through comments and docstrings.
\item
  Common sources of error (e.g.~bad user input) are handled in a
  sensible fashion.
\item
  Outputs are well laid out and (for figures) visually pleasing.
\item
  The student displays knowledge of a broad range of Python
  capabilities, as apropriate to the task.
\item
  The student is able to explain their code well, perhaps showing some
  awareness of alternative ways in which the problem could have been
  tackled.
\end{itemize}

An \textbf{excellent mark} will be awarded where:

\begin{itemize}
\item
  The solution fully accomplishes all aspects of the task without error.
\item
  Code is clear, concise and readily-understood by anyone who reads it.
\item
  Code is fully-documented with comprehensive docstrings and comments
  that help the reader understand how the code works.
\item
  The code is robust, with all common sources of error handled elegantly
  and effectively.
\item
  Figures are of a high standard, with care taken to maximise clarity
  and information content.
\item
  The student displays full knowledge of the Python programming language
  (to the extent covered in the course), and consistently chooses the
  most appropriate tools at their disposal.
\item
  The student has a comprehensive understanding of their code, and is
  able to discuss the advantages and disadvantages of various solution
  strategies, justifying why they chose to adopt one and not the others.
\item
  The solution may go beyond what is strictly required to answer the
  assigned task, or employ aspects of Python that were not explored
  during the course.
\end{itemize}

Good luck!

\begin{center}\rule{0.5\linewidth}{\linethickness}\end{center}

    \hypertarget{the-fibonacci-sequence-10}{%
\subsection*{1. The Fibonacci Sequence
(10\%)}\label{the-fibonacci-sequence-10}}

The \href{https://en.wikipedia.org/wiki/Fibonacci_number}{Fibonacci
seqence} is a famous sequence of numbers. Each term in the sequence is
obtained by adding together the previous two terms. The first two terms
are defined to be (1, 1, \ldots{}).

Write a function that will return a list containing the first \(n\)
terms of a Fibonacci sequence.

    \hypertarget{the-ideal-gas-law-10}{%
\subsection*{2. The ideal gas law (10\%)}\label{the-ideal-gas-law-10}}

The ideal gas law relates pressure (\(P\)), volume (\(V\)) and
temperature (\(T\)) for an `ideal' gas. It states that

\[ PV = nRT \]

where \(n\) is the number of `moles' of gas present, and \(R\) is a
constant,
\(R = 8.3145\,\mathrm{L}\,\mathrm{kPa}\,\mathrm{mol}^{-1}\,\mathrm{K}^{-1}\).
Write a function that allows the user to specify \textbf{any three} of
the quantities \(P\), \(V\), \(n\) and \(T\), and obtain the fourth.

    \hypertarget{the-substitution-cipher-20}{%
\subsection*{3. The substitution cipher
(20\%)}\label{the-substitution-cipher-20}}

In class, we encountered the Caesar cipher for encoding messages.
Another kind of cipher is a
`\href{https://en.wikipedia.org/wiki/Substitution_cipher}{substitution
cipher}'. To encode a message, we first choose a `keyword' - for
example, \texttt{CHEESE}. If any letter appears in the keyword more than
once, we delete the subsequent occurrences (giving \texttt{CHES}).

Next, we write out a full alphabet, and a second version where the
characters that appear in the keyword are shifted to the front:

\begin{verbatim}
A B C D E F G H I J K L M N O P Q R S T U V W X Y Z    <== Plaintext
C H E S A B D F G I J K L M N O P Q R T U V W X Y Z    <== Cipher
\_____/    | |   |                   |    
Keyword    Characters in keyword moved to front
\end{verbatim}

To encode a message, we look up each character in the `plaintext' row
and replace it with the corresponding character from the `cipher' row.
So, using our keyword \texttt{CHEESE}, the message
\texttt{I\ AM\ WORKING\ ON\ MY\ ASSIGNMENT} becomes
\texttt{G\ CL\ WNQJGMD\ NM\ LY\ CRRGDMLAMT}. To decode a message, we
look up characters in the `cipher' row, and replace them with the
corresponding character from the `plaintext' row. Any characters not in
the alphabet (e.g.~punctuation characters and spaces) should be
unchanged by the encoding/decoding operation.

Write a function which can encode and decode messages using any keyword.
To help you check that it is working properly, here is a message that
has been encoded using the keyword \texttt{PYTHON}:
\texttt{VOFF\ HJIO,\ XJS\ BPUO\ QSTTOQQNSFFX\ HOTJHOH\ P\ GOQQPAO!}

    \hypertarget{global-seismicity-20}{%
\subsection*{4. Global seismicity (20\%)}\label{global-seismicity-20}}

\textbf{This question has two parts; you should attempt both.}

The file \texttt{earthquakes.dat} contains information about earthquakes
that have occurred in the last 10 years, drawn from the
\href{http://www.globalcmt.org}{Global CMT catalogue}. The file contains
columns corresponding to the earthquake location, magnitude and
date/time, as follows:

\begin{longtable}[]{@{}llllllllll@{}}
\toprule
Latitude & Longitude & Depth & Magnitude & Year & Month & Day & Hour &
Mins & Secs\tabularnewline
\midrule
\endhead
36.87 & 69.95 & 54.3 & 5.1 & 2000 & 01 & 01 & 05 & 24 &
35.3\tabularnewline
-60.72 & 153.67 & 10.0 & 6.0 & 2000 & 01 & 01 & 05 & 58 &
19.8\tabularnewline
\ldots{} & \ldots{} & \ldots{} & \ldots{} & \ldots{} & \ldots{} &
\ldots{} & \ldots{} & \ldots{} & \ldots{}\tabularnewline
\bottomrule
\end{longtable}

\begin{enumerate}
\def\labelenumi{\arabic{enumi}.}
\item
  Make a global map plotting the location of every earthquake in the
  file.
\item
  Write a function that allows the user to specify ranges for any or all
  of: latitude, longitude and depth. The function should then

  \begin{enumerate}
  \def\labelenumii{\arabic{enumii}.}
  \tightlist
  \item
    produce a map of only those events that fall within the specified
    range(s), with information about earthquake depth shown using
    colour, AND
  \item
    write out a new file (in the same format as
    \texttt{earthquakes.dat}) that contains only the selected events.
  \end{enumerate}
\end{enumerate}

    \hypertarget{infrared-spectroscopy-20}{%
\subsection*{5. Infrared Spectroscopy
(20\%)}\label{infrared-spectroscopy-20}}

The file \texttt{11octrll32bs1\_2016-10-11T12-48-25.asp} is produced by
an infrared spectrometer. Look at the file contents. It is formatted as
follows:

\begin{itemize}
\tightlist
\item
  Line 1: Number of data points in file
\item
  Line 2: Wavenumber (cm\({}^{-1}\)) associated with first measurement
  in file
\item
  Line 3: Wavenumber associated with last measurement in file
\item
  Line 4-6: Miscellaneous information (not required for this exercise)
\item
  Line 7-end: Reflectance measurements (\%).
\end{itemize}

You can assume that the measurements are evenly-spaced in wavenumber.

The filename encodes some additional information:

\begin{verbatim}
11octrll32bs1_2016-10-11T12-48-25.asp
\___/\____/\| \________/ \______/
  A     B   C      D        E
  
A - Date of experiment (11 October)
B - Location of experiment ('rll32b')
C - Sample number (Sample 1) -- starting with 's'
D - Date of experiment (again; 11 October 2016)
E - Time of experiment (12:48:25)
\end{verbatim}

Write a function that takes the filename as argument, and produces a
high-quality figure, showing reflectance (y-axis) plotted against
wavenumber (x-axis). The following information should also be displayed
on the figure: - The date and time at which the experiment was run - The
location, panel and sample numbers - The filename of the datafile.

Your function should work for \emph{any} file of this format. To help
you test this, you have been provided with a second file from the same
instrument - \texttt{12octrll61s1\_2016-10-12T10-36-43.asp}.

    \hypertarget{tic-tac-toe-20}{%
\subsection*{6. Tic-Tac-Toe (20\%)}\label{tic-tac-toe-20}}

\href{https://en.wikipedia.org/wiki/Tic-tac-toe}{Tic-tac-toe} (or
`noughts and crosses') is a game for two players. Players take it in
turns to place their mark (`X' for player 1, `O' for player 2) in an
empty space on a 3 x 3 grid. A player getting 3 marks in a line (row,
column or diagonal) wins.

Write code to allow you to play Tic-Tac-Toe against the computer. The
computer should play randomly, placing its mark in any empty cell.


    % Add a bibliography block to the postdoc
    
    
    
    \end{document}
